% --- UNIVERSAL PREAMBLE BLOCK (Adapted for Beamer Poster) ---
\documentclass[final]{beamer}
\usepackage[orientation=portrait,size=a0,scale=1.0]{beamerposter}
% Keep font setup TeX-live portable (avoid requiring system fonts).
\usepackage{lmodern}
\usepackage[english]{babel}
\usepackage[protrusion=true,expansion=false]{microtype}
\usepackage{booktabs}
\usepackage{pgfplots}
\pgfplotsset{compat=1.18}

% Layout and Appearance
\usetheme{Berlin}
\usecolortheme{whale}
\setbeamertemplate{navigation symbols}{}
\setbeamertemplate{caption}[numbered]
\definecolor{TuebingenBlue}{RGB}{0, 86, 158}
\definecolor{TuebingenBlueLight}{RGB}{40, 120, 190}
\setbeamercolor{footlinebar}{bg=TuebingenBlue,fg=white}
\setbeamercolor{footlinebarlight}{bg=TuebingenBlueLight,fg=white}
\setbeamertemplate{footline}{
    \begin{beamercolorbox}[wd=\paperwidth,ht=0.55cm,dp=0cm]{footlinebar}
    \end{beamercolorbox}
    \begin{beamercolorbox}[wd=\paperwidth,ht=0.25cm,dp=0cm]{footlinebarlight}
    \end{beamercolorbox}
}

% Fonts - LARGER SIZES
\usefonttheme{professionalfonts}

% Colors for custom boxes
\definecolor{TuebingenRed}{RGB}{165, 30, 55}
\definecolor{DarkSlate}{RGB}{47, 79, 79}
\setbeamercolor{block title}{bg=TuebingenRed,fg=white}
\setbeamercolor{block body}{bg=white,fg=black}
\setbeamercolor{alerted text}{fg=TuebingenRed}
\setbeamercolor{tldrbox}{bg=gray!10,fg=black}
\setbeamercolor{tldrtitle}{bg=TuebingenRed,fg=white}
\setbeamerfont{block title}{size=\LARGE}
\setbeamerfont{block body}{size=\large}
\setbeamerfont{itemize/enumerate body}{size=\large}
\setbeamerfont{itemize/enumerate subbody}{size=\large}

% Contextual block style for top/bottom rows
\setbeamercolor{block title contextual}{bg=DarkSlate,fg=white}
\setbeamercolor{block body contextual}{bg=white,fg=black}

% Graphicx
\usepackage{graphicx}
\usepackage{tikz}
\usetikzlibrary{shapes,arrows,positioning}

% --- POSTER CONTENT ---

\title{\Huge \textbf{A General Pipeline for Genome Mining of Peptides and Enzymes Using Protein Language Models}}
\subtitle{\LARGE Lasso peptide precursors, $\beta$-lactamases, and conserved alt-frame micropeptides from all-ORF search spaces}
\author{\large \textbf{Magnus Ohle} \and Nadine Ziemert}
\institute{\large Translational Genome Mining for Natural Products, Universität Tübingen}
\date{\today}

\begin{document}

\begin{frame}[t]

% --- LOGO ABOVE TITLE ---
\vspace*{0.01\paperheight}
\begin{flushleft}
    \hspace{1cm}\includegraphics[height=4cm]{uni-tuebingen.png}
\end{flushleft}
\vspace{0.3cm}

% --- HEADER BLOCK ---
\begin{beamercolorbox}[wd=\paperwidth,sep=0.5cm,center]{headline}
    \usebeamerfont{title}\inserttitle\par
    \vspace{0.4em}
    \usebeamerfont{subtitle}\insertsubtitle\par
    \vspace{0.8em}
    \usebeamerfont{author}\insertauthor\par
    \vspace{0.3em}
    \usebeamerfont{institute}\insertinstitute
\end{beamercolorbox}
\vspace{0.25cm}

% ============================================================================
% TOP ROW: TL;DR / Briefing / Methods (Contextual)
% ============================================================================
\begingroup
\setbeamercolor{block title}{use=block title contextual,bg=block title contextual.bg,fg=block title contextual.fg}

\begin{columns}[T]

\begin{column}{0.32\paperwidth}
    \begin{beamercolorbox}[wd=\linewidth, sep=1em, rounded=true, shadow=true]{tldrbox}
        \begin{beamercolorbox}[wd=\dimexpr\linewidth-2em\relax, sep=0.4em, center]{tldrtitle}
            \textbf{\Huge TL;DR}
        \end{beamercolorbox}
        \vspace{0.6em}
        {\large
        \begin{itemize}
            \item \textbf{Take-home:} Exhaustive ORF enumeration + frozen protein language model embeddings enables \textbf{annotation-free retrieval} of peptides and enzymes from genomes.
            \item \textbf{All-ORF:} 6-frame translation with length filters (20--120 aa lasso; 200--400 aa $\beta$-lactamase) + frozen ESM2.
            \item \textbf{Generalizable:} Same pipeline for lassos and $\beta$-lactamases.
            \item \textbf{Performance:} 92\% lasso Top-50 recall; 100\% Top-5 $\beta$-lactamase recall.
            \item \textbf{Discovery:} Alt-frame loci with conservation beyond a synonymous null ($z > 10$).
        \end{itemize}
        }
    \end{beamercolorbox}
\end{column}

\begin{column}{0.32\paperwidth}
    \begin{block}{Briefing: The Concept}
        {\large
        \textbf{Background:}
        \textbf{Lasso peptides} are ribosomally synthesized natural products (RiPPs) whose mature peptide forms a \textbf{lariat-knot} topology; they originate from short precursor peptides (leader + core).
        \textbf{Genome mining} searches genomes for biosynthetic genes and precursors to discover new natural products.

        \vspace{0.4em}
        \textbf{Main challenge:} Short and alternative ORFs are often \textbf{missed or mis-annotated}, and family-specific models (e.g. HMMs) can miss \textbf{divergent} variants.

        \vspace{0.4em}
        \textbf{Aim / hypothesis:} A frozen protein language model provides a \textbf{family-agnostic similarity signal} that enables retrieval of true targets from an \textbf{exhaustive all-ORF} search space.

        \vspace{0.4em}
        \textbf{Proposed solution:} \textit{Embedding-based Retrieval from an All-ORF Search Space}
        \begin{itemize}
            \item \textbf{Input:} Raw DNA $\rightarrow$ 6-frame ORF enumeration.
            \item \textbf{Engine:} Frozen ESM2-8M/35M embeddings.
            \item \textbf{Query:} Cosine similarity to seed sequences.
        \end{itemize}
        }
    \end{block}
\end{column}

\begin{column}{0.32\paperwidth}
    \begin{block}{Approach\vphantom{g}}
        {\large
        \textbf{Pipeline}
        \begin{enumerate}
            \item \textbf{Extraction:} Window $\pm$20kb or full genome.
            \item \textbf{Embedding:} Mean-pooled ESM2.
            \item \textbf{Ranking:} Top-$N$ mean cosine similarity (consensus over multiple nearest hits) to reduce single-hit noise.
        \end{enumerate}

        \vspace{0.4em}
        \textbf{Proposed solution (one line):} enumerate all ORFs $\rightarrow$ embed with ESM2 $\rightarrow$ rank by similarity to validated seeds (no task-specific training).
        
        \vspace{0.5em}
        \textbf{Validation}
        \begin{itemize}
            \item \textbf{Lasso:} 255 loci, 1,869 genomes (20--120 aa filter).
            \item \textbf{Beta-lac:} 5 full genomes (23k--45k ORFs per genome; 200--400 aa filter).
            \item \textbf{Alt-frame:} 88 loci, synonymous null ($N=200$).
        \end{itemize}
        }
    \end{block}
\end{column}

\end{columns}
\endgroup

% ============================================================================
% MIDDLE ROW: Act I / Act II (Results - Red, Center of Attention)
% ============================================================================
\begin{columns}[T]

\begin{column}{0.48\paperwidth}
    \begin{block}{Semantic Retrieval: Lasso Peptides}
        {\large \textbf{Test:} Retrieve $\sim$40 aa precursors from a $\sim$40 kb all-ORF search space (20--120 aa; 255 loci).}

        \vspace{0.3em}
        {\large \textbf{What you see:} Top-k recall = fraction of loci where the known precursor ORF ranks within the top $k$ among all candidate ORFs in the window.}\\
        {\large \textbf{Take-home:} Embedding retrieval can recover short, diverse RiPP precursors despite large all-ORF background noise.}
        
        \vspace{0.5em}
        \begin{center}
        \begin{tikzpicture}
        \begin{axis}[
            width=0.85\linewidth, height=10.5cm,
            ybar, ymin=0, ymax=1,
            ytick={0,0.25,0.5,0.75,1.0}, yticklabels={0\%,25\%,50\%,75\%,100\%},
            symbolic x coords={Top-1,Top-5,Top-10,Top-50},
            xtick=data, bar width=24pt,
            axis x line*=bottom, axis y line*=left,
            nodes near coords,
            nodes near coords style={font=\Large},
            tick label style={font=\large},
        ]
            \addplot[fill=TuebingenRed!70, draw=TuebingenRed] coordinates {
                (Top-1,0.475) (Top-5,0.627) (Top-10,0.824) (Top-50,0.918)
            };
        \end{axis}
        \end{tikzpicture}
        \end{center}
        \vspace{0.3em}
        {\large \textbf{Result:} $\sim$92\% recall in Top-50. Embedding retrieval performs well on short, variable RiPP precursors.}\\
        \vspace{0.2em}
        {\small \textit{Model comparison:} ESM2-8M $\rightarrow$ 35M improves retrieval; 150M shows degraded performance in the all-ORF setting.}
    \end{block}
    
    \vspace{0.6cm}
    
    \begin{block}{Semantic Retrieval: $\beta$-lactamases}
        {\large \textbf{Stress Test:} Retrieve a known enzyme from full-genome ORF enumeration (200--400 aa; 23k--45k ORFs).}

        \vspace{0.3em}
        {\large \textbf{What you see:} Holdout by accession; rank all 200--400 aa ORFs extracted from full genomes.}\\
        {\large \textbf{Take-home:} The same retrieval engine generalizes from short precursors to enzyme-sized proteins without re-training.}
        
        \vspace{0.5em}
        \begin{beamercolorbox}[wd=\linewidth, rounded=true, shadow=false, sep=0.6em, center]{block body}
            \textbf{\LARGE Result: 100\% Top-5 Recall (5/5 Genomes)}
        \end{beamercolorbox}
        
        \vspace{0.5em}
        \begin{center}
        \begin{tikzpicture}
        \begin{axis}[
            width=0.85\linewidth, height=10.5cm,
            ybar, ymin=0, ymax=1,
            ytick={0,0.25,0.5,0.75,1.0}, yticklabels={0\%,25\%,50\%,75\%,100\%},
            symbolic x coords={Top-1,Top-5,Top-10,Top-50},
            xtick=data, bar width=24pt,
            axis x line*=bottom, axis y line*=left,
            nodes near coords,
            nodes near coords style={font=\Large},
            tick label style={font=\large},
        ]
            \addplot[fill=TuebingenRed!70, draw=TuebingenRed] coordinates {
                (Top-1,0.8) (Top-5,1.0) (Top-10,1.0) (Top-50,1.0)
            };
        \end{axis}
        \end{tikzpicture}
        \end{center}
        {\large \textbf{Implication:} Family-agnostic retrieval generalizes without task-specific training.}
    \end{block}
\end{column}

\begin{column}{0.48\paperwidth}
    \begin{block}{ORF evaluation: Alt-Frame Micropeptides}
        {\large \textbf{Question:} Are overlapping-frame peptides under selective constraint?}
        \vspace{0.3em}
        
        {\large \textbf{Method:} ORF conservation vs. \textbf{Synonymous Null}.
        \begin{itemize}
            \item Preserve main CDS amino acids.
            \item Shuffle synonymous codons (N=200).
            \item If alt-frame conservation exceeds null $\Rightarrow$ evidence of constraint.
        \end{itemize}
        }

        \vspace{0.8em}
        {\large \textbf{What you see:} Each point is one locus; $x$ = expected identity under synonymous-codon shuffling (null), $y$ = observed identity among surviving (stop-free) peptides.}\\
        {\large \textbf{Take-home:} Loci above the diagonal (red, $z>10$) show alt-frame conservation beyond what is expected from constraints on the primary CDS alone.}

        \vspace{0.4em}
        \begin{center}
        \begin{tikzpicture}
        \begin{axis}[
            width=0.9\linewidth, height=21cm,
            xlabel={Null identity mean}, ylabel={Observed identity},
            xmin=0, xmax=1, ymin=0, ymax=1,
            xtick={0, 0.25, 0.5, 0.75, 1.0},
            ytick={0, 0.25, 0.5, 0.75, 1.0},
            grid=both, grid style={line width=.1pt, draw=gray!30},
            legend pos=south east,
            label style={font=\large},
            tick label style={font=\large},
            legend style={font=\large},
        ]
            \addplot[only marks, mark=*, mark size=4pt, color=gray!50]
                table [x=null_identity_mean, y=obs_identity, col sep=tab] {poster_assets/altframe_scatter_all.tsv};
            \addplot[only marks, mark=*, mark size=5pt, color=TuebingenRed]
                table [x=null_identity_mean, y=obs_identity, col sep=tab] {poster_assets/altframe_scatter_sig.tsv};
            \legend{all loci, $z>10$ loci}
        \end{axis}
        \end{tikzpicture}
        \end{center}
        
        \vspace{0.5em}
        {\large \textbf{Finding:} 32/88 loci show strong alt-frame conservation ($z > 10$; $z$ computed from the null identity distribution).
        \begin{itemize}
            \item \textbf{phnU} ($z=70.18$), \textbf{cyoA} ($z=30.30$), and \textbf{ceoR} ($z=27.85$) are extreme outliers.
            \item Consistent with unannotated micropeptides.
        \end{itemize}
        }

        \vspace{0.4em}
        {\small \textit{Complementary signal:} A genome-wide ORF index can simply count how often identical peptides recur across genomes or BGC regions, acting as a fast frequency-based triage without null permutations.}
    \end{block}
\end{column}

\end{columns}

\vspace{0.25cm}

% ============================================================================
% BOTTOM ROW: Discussion / Outlook / Conclusion (Contextual)
% ============================================================================
\begingroup
\setbeamercolor{block title}{use=block title contextual,bg=block title contextual.bg,fg=block title contextual.fg}

\begin{columns}[T]

\begin{column}{0.32\paperwidth}
    \begin{block}{Discussion: Implications\vphantom{g}}
        {\large
        \begin{itemize}
            \item \textbf{Simple, Scalable Prior:} Exhaustive all-ORF enumeration + frozen PLM embeddings retrieves short RiPPs and larger enzymes without complex HMMs.
            \item \textbf{Model Efficiency:} ESM2-8M/35M perform strongly for retrieval, with a sharper similarity landscape than larger models.
            \item \textbf{Altframes:} The codon-preserving null suggests specific alt-frame ORFs (e.g., \textit{phnU}, \textit{ceoR}) show conservation consistent with functional micropeptides.
        \end{itemize}
        }
    \end{block}
\end{column}

\begin{column}{0.32\paperwidth}
    \begin{block}{Outlook\vphantom{g}}
        {\large
        \begin{itemize}
            \item \textbf{General Engine:} Applicable to any family given validated precursors; model limitations should be evaluated per target.
            \item \textbf{Systematic Mining:} Scale ORF indexing for rapid cross-genome queries and candidate triage.
            \item \textbf{Targeted Validation:} Prioritize high-scoring alt-frame candidates and ambiguous $\beta$-lactamase ORFs for follow-up.
        \end{itemize}
        }
    \end{block}
\end{column}

\begin{column}{0.32\paperwidth}
    \begin{block}{Conclusion\vphantom{g}}
        {\large
        This project demonstrates that frozen protein language models, applied to exhaustive ORF enumeration, can retrieve diverse gene families without task-specific training.
        This makes genome mining less dependent on perfect annotation and family-specific models, and it surfaces candidate biology that is easy to miss (short ORFs, alt-frame peptides).
        
        \vspace{0.8em}
        \textbf{Code and all results:} \texttt{github.com/Mvgnu/Lasso}
        }
    \end{block}
\end{column}

\end{columns}
\endgroup

\end{frame}
\end{document}
